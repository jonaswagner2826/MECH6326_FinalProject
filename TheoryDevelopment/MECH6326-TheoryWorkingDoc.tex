% Standard Article Definition
\documentclass[9pt, onecolumn]{report}

% Page Formatting
\usepackage[margin=0.5in]{geometry}

% \setlength\parindent{0pt}
% \pagestyle{fancy}

% Graphics
\usepackage{graphicx}
\usepackage{xcolor}

% Math Packages
\usepackage{physics}
\usepackage{amsmath, amsfonts, amssymb, amsthm}
\usepackage{mathtools}

% Extra Packages
% \usepackage{pdfpages}
% \usepackage{hyperref}
\usepackage{todonotes}
% \usepackage{listings}

%Custom Commands
\newcommand{\N}{\mathbb{N}}
\newcommand{\Z}{\mathbb{Z}}
\newcommand{\Q}{\mathbb{Q}}
\newcommand{\R}{\mathbb{R}}
\newcommand{\C}{\mathbb{C}}

% \newcommand{\SigAlg}{\mathcal{S}}

% \newcommand{\Rel}{\mathcal{R}}

% \newcommand{\toI}{\xrightarrow{\textsf{\tiny I}}}
% \newcommand{\toS}{\xrightarrow{\textsf{\tiny S}}}
% \newcommand{\toB}{\xrightarrow{\textsf{\tiny B}}}

% \newcommand{\divisible}{ \ \vdots \ }
% \newcommand{\st}{\ : \ }

% % Theorem Definition
% \newtheorem{definition}{Definition}
% \newtheorem{assumption}{Assumption}
% \newtheorem{theorem}{Theorem}
% \newtheorem{lemma}{Lemma}
% \newtheorem{proposition}{Proposition}
% \newtheorem{remark}{Remark}
% % \newtheorem{example}{Example}
% % \newtheorem{counterExample}{Counter Example}


%opening
\title{
    MECH 6326 - Optimal Control and Dynamic Programming \\ 
    Final Project Working Doc
}
\author{Alyssa Vellucci and Jonas Wagner}
% \date{2023, February 24\textsuperscript{th}}

\begin{document}

\maketitle

% \section{Proposal Things}
% \section*{D\&D Explanation}
% The multi-step decision making under uncertainty problem we will be analyzing is that of combat during a tabletop roleplaying game (TTRPG), specifically a combat encounter under the ruleset from Dungeons \& Dragons fifth edition (D\&D 5e).

% D\&D consists of a group or party of players who go on an adventure together.
% The party is composed of multiple playable characters (PCs) that exist within an environment overseen by a Dungeon Master (DM).
% In a typical D\&D combat, the players will fight a creature (or multiple creatures) being controlled by the DM. 
% The players have certain skills their characters can use, and their effectiveness is determined by a dice roll. 
% The creatures overseen by the DM also have certain skills or tactics they can use against the players, and that efficacy is also determined by a dice roll; however the outcomes of these rolls are hidden from the players. 
% % Simplistically this process can be modeling as a closed loop system where the current state of the battle can be used to inform the player's decision-making and ideally used for a successful combat encounter.

% \section*{System Model}
% The combat encounter will be simplified to fit the scope of the project, but the full combat system is very well documented in the D\&D 5e ruleset; thus, the system is simplistically defined as the environment the combat takes place in. The PCs act as controllers in this environment. All rules governing the actions of the players, enemies, and environment are dictated by the D\&D 5e ruleset. 

% \subsection*{System States:}
% The states of the system include the following for both the PC and enemy creature: 
% (i) hit points (HP), 
% (ii) position within the environment, 
% (iii) condition modifiers (ex. Blinded, prone, poisoned, etc.),
% and (iv) potentially the spell slots remaining (how many times a PC or creature can perform an ability).

% \subsection*{System Inputs:}
% The control inputs would include all the actions the PCs perform.
% This can include a PC conducting an attack, moving a set distance, casting a utility spell, or interacting with an object in their environment. 
% This will be simplified to selecting between a finite number possible actions (melee/ranged attack, cast attack/utility spell, etc.), alongside deciding if and where to move, and potentially another selection between the more restrictive bonus actions. Each action should further the goal of the performance objective.

% \subsection*{Stochastic Disturbances:}
% The stochastic uncertainty within the system comes from the dice rolls made when performing an action.
% Generally a threshold must be met to yield a success along and additional dice rolls to determine effectiveness (i.e. damage) of an action.
% Each roll is also be modified by the specific attributes of the PCs and creatures.

% Since the players don't directly know the stats (armor class, HP totals, what abilities they can use, resistances and weaknesses, etc.) of the enemies they face, additional dice rolls may be taken to determine how effectively the enemies stats are known.
% The initial simplification will eliminate this uncertainty.

% \section*{Primary Control Objective:}
% The primary control objectives of both the PCs and creatures will be to maximize their own HP while minimizing the opponent's HP.
% The objective could also be modified to work with other members of the party, environmental considerations (dangerous/advantageous terrain/formation etc.), and/or increase/decrease the overall length of the encounter.

% \section*{Potential Results:}
% The potential results of this analysis would be the ideal order of actions a player would take to have the optimal combat encounter. This could be displayed as a trajectory of the HP totals of the PCs and the creatures at each time step (combat round). With these graphs, we could directly compare control laws with different parameters (conditions affecting PCs, movement penalties, etc.) and the cost function could also be modified and compared for different scenarios: shortest combat encounter, a chase/fleeing combat encounter, or a protect-the-NPC encounter.

% Another result of the problem could involve a 1D or 2D visualization of the PC and creature movement as it evolves over the course of the combat. The movement mechanic is important in D\&D combat, and rarely do two combatants stand in front of one another the entire encounter.

\chapter{Simple System Model}
\section{System Definition}
\subsection{Assumptions:}
\begin{itemize}
    \item Movement: Single movement per turn
    \begin{itemize}
        \item Deterministic
        \item 1 square movement
        \item move then action
    \end{itemize}
    \item Actions: Single action per time step
    \begin{itemize}
        \item Melee (hit check)(d6) - Short range
        \item Ranged (hit check)(d8) - Longer range
        \item Health Potion (d4 + 1)
        \item Nothing
    \end{itemize}
    \item Characters
    \begin{itemize}
        \item 1 PC and 1 Monster
        \item Identical Specs/modifiers
        \item Same weapon (+2) ??????????????????? (different between melee and ranged?)
    \end{itemize}
    \item Monster
    \begin{itemize}
        \item Monster move in standard pattern
        \item Monster cannot heal
    \end{itemize}
    \item Infinite Time Horizon
    \item Infinite Battlefield and no Obstacles
\end{itemize}

\subsection{Environment Definition}
\subsubsection{States}
Let each character be associated with position and HP states.
For position, let \[
    x_{pc,p}, x_{mn,p} \in \mathcal{X}_{p} \subseteq \Z^{2}
\] describe the position on an infinite 2-d grid.
For HP, let \[
    x_{pc,hp}, x_{mn,hp} \in \mathcal{X}_{hp} \subseteq \Z_+ = \{0, 1, 2, \dots\}
\] describe the HP for each character.

\subsubsection{Inputs}
The inputs to the system consist of movement and actions impacting the position and hp states respectively.
For movement, a deterministic input of \[
    u_{pc,m}, u_{mn,m} \in \mathcal{U}_{m} = \{(-1,0), (+1,0), (0,-1),(0,+1), {\color{red} (-1,-1), (-1,+1), (+1,-1),(+1,+1)}\}
\]

For actions, all the actions (except nothing) each are stochastic and can be represented as Markov chains or as a combination of input and noise term, $u_{pc,a}, u_{mn,a} \in \mathcal{U}_{a} = \{\text{Melee, Ranged, Heal, Nothing}\}$.

For Melee and Ranged attacks, the character acts upon another character's HP where the impact on HP is as follows:
\begin{enumerate}
    \item Ensure in range for either melee or ranged attack - otherwise can't attack.
    \item "Roll" for success/fail - if fail then self-loop on opponent HP
    \item "Roll" for effectiveness - opponent HP decreased by Weapon/self Modifiers (2) + d6/d8
\end{enumerate}

The PC is allowed to use a health potion which has a stochastic effect upon the player's health:
\begin{enumerate}
    \item {\color{red} Ensure potion is available - otherwise can't heal}
    \item "Roll" for effectiveness - player's HP increased by health modifier (1) + d4
\end{enumerate}

\subsubsection{Problem Statement}
For the simplistic case, let states at time-step $k$, be \[
    x_k = \mqty[x_{pc,p}\\x_{mn,p}\\x_{pc,hp}\\x_{mn,hp}] \in \mathcal{X} = \mathcal{X}_p^2 \cross \mathcal{X}_{hp}^2 \subseteq \Z^{4} \cross \Z_{+}^{2}
\] Let the inputs to the system be only the players inputs \[
    u_k = \mqty[u_{pc,p}\\u_{pc,a}] \in \mathcal{U} = \mathcal{U}_{p} \cross \mathcal{U}_{a} %\subseteq \{(-1,0), (+1,0), (0,-1),(0,+1), {\color{red} (-1,-1), (-1,+1), (+1,-1),(+1,+1)}\} \cross \{\text{Melee, Ranged, Heal, Nothing}\}
\] The monster's inputs to the system will be incorporated as a deterministic and stochastic input that are closed-loop within the system and treated as part of the nonlinear aspects of the update function/Markov chains.

The evolution of the system can be described as Markov chains or by a nonlinear update function:\[
    x_{k+1} = f(x_k, u_k)
\] which is described as ... (more complicated..)










% Appendix ----------------------------------------------
% \newpage
% \appendix
% \bibliographystyle{plain}
% \bibliography{refs.bib}




\end{document}
